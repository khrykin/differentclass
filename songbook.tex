% XeLaTeX can use any Mac OS X font. See the setromanfont command below.
% Input to XeLaTeX is full Unicode, so Unicode characters can be typed directly into the source.

% The next lines tell TeXShop to typeset with xelatex, and to open and save the source with Unicode encoding.

%!TEX TS-program = xelatex
%!TEX encoding = UTF-8 Unicode

\documentclass[14pt]{article}
%\usepackage[chordbk]{songbook} 
\usepackage{gchords}  
\usepackage{geometry}
\usepackage{rotating}
\usepackage{multicol}
\usepackage{enumitem,xcolor}

\geometry{a4paper}                   % ... or a4paper or a5paper or ... 
%\geometry{landscape}                % Activate for for rotated page geometry
%\usepackage[parfill]{parskip}    % Activate to begin paragraphs with an empty line rather than an indent
\usepackage{graphicx}
\usepackage{amssymb}

% Will Robertson's fontspec.sty can be used to simplify font choices.
% To experiment, open /Applications/Font Book to examine the fonts provided on Mac OS X,
% and change "Hoefler Text" to any of these choices.

\usepackage{fontspec,xltxtra,xunicode}

\renewcommand\thesection{}

\defaultfontfeatures{Mapping=tex-text}
\setromanfont[Mapping=tex-text]{Hoefler Text}
\setsansfont[Scale=MatchLowercase,Mapping=tex-text]{Gill Sans}
\setmonofont[Scale=MatchLowercase]{Andale Mono}
\title{Different Class}
\author{Songbook}
\date{}                                           % Activate to display a given date or no date
% Chords 

% C
\newcommand{\CMaj}{\chord{t}{x, p3, p2, o, p1, o}{C}}
\newcommand{\CMajSeven}{\chord{t}{x, p3, p2, o, o, o}{CMaj7}}
\newcommand{\CSeven}{\chord{t}{x, p3, p2, p3, p1, o}{C7}}

% D

\newcommand{\DMaj}{\chord{t}{x, x, o, p2, p3, p2}{D}}
\newcommand{\DSeven}{\chord{t}{x, x, o, p2, p1, p2}{D7}}

% E

\newcommand{\EMaj}{\chord{t}{o, p2, p2, p1, o, o}{E}}



% F#

\newcommand{\Fshm}{\chord{t}{p2, p4, p4, p2, p2, p2}{F\#m}}


%G
\newcommand{\GMaj}{\chord{t}{p3, p2, o,o, o, p3}{G}}
\newcommand{\GMajSeven}{\chord{t}{p3, p2, o,o, o, p2}{GMaj7}}
\newcommand{\GSeven}{\chord{t}{p3, p2, o,o, o, p1}{G7}}

%A

\newcommand{\AMaj}{\chord{t}{x, o, p2, p2, p2, o}{A}}

\newcounter{SongVerseCnt}

\newenvironment{myenumerate}{%
  \edef\backupindent{\the\parindent}
  \begin{enumerate}[label=\color{gray}\theenumi]
  \setlength{\parindent}{\backupindent}
}{\end{enumerate}}


\newenvironment{song}[2]
{
	\setcounter{SongVerseCnt}{0}
	%\keyIndex{#1}{\theSBSongCnt}

		\begin{center}
			\section{#1}
			#2
		\end{center}
	\begin{multicols}{2}
\normalsize
\begin{myenumerate}
}{
\end{myenumerate}
\end{multicols}
\newpage
}


\definecolor{blue}{rgb}{0.2, 0.5, 0.5}

\newenvironment{SongVerse}
{
	%\stepcounter{SongVerseCnt}
	\item
	\setlength{\parindent}{0cm}
}{
\newcommand{\SBChordRaise}{2.25ex}
}
\newcommand{\ch}[2]{
	\makebox[0pt][l]{\raisebox{2.5ex}{\color{blue}\sffamily \,\,#1}}
	\makebox{#2}
}



% \def\numfrets{4}

\begin{document}

\maketitle

\newpage

\tableofcontents 
\newpage

\begin{song}{Mis-Shapes}{

    \git{Verse}

	\mbox{ \AMaj \EMaj \Fshm \DMaj \DSeven}

    \git{Chorus}

	\mbox{ \GMaj \GMajSeven \GSeven \CMaj \CMajSeven \CSeven }
	

	\mbox{ \Em \EmSix \EmaddC }

}

	 \begin{SongVerse}
		\ch{A}{Mis-shapes}, mistakes, misfits. 

		\ch{E}{Raised} on a diet of broken biscuits, oh \ch{F\#m}{}

		We don't look the same as you \ch{D}{}
		
		We don't do the things you do,
		
		But \ch{D7}{we} live around here too, oh really. 
	 \end{SongVerse}

	 \begin{SongVerse}
		\ch{A}{Mis-shapes}, mistakes, misfits,

		We'd \ch{E}{like} to go to town but we can't risk it, oh \ch{F\#m}{} 

		'Cos they just want to keep us out. \ch{D}{} 
		
		You could end up with a smack in the mouth

		\ch{D7}{Just} for standing out, now really.
	 \end{SongVerse}

	 \begin{SongVerse}

		\ch{A}Brothers, sisters, can't you \ch{E}see? 

		The future's owned by you and \ch{F\#m}me. 

		There won't be fighting in the \ch{D}street. 

		They think they've got us beat, 

		But \ch{D\up7}revenge is going to be so sweet. 
	\end{SongVerse}

	\begin{SongVerse}
		
		\ch{G} \quad We're making a \ch{G\up{maj7}}move, 

		we're making it \ch{G\up 7}now. 

		We're coming out of the sidelines. 

		\ch{C}\quad Just put your \ch{C\up{maj7}}hands up -- it's a \ch{C\up7}raid \ch{C}... \ch{C\up7}yeah.

		We want your \ch{Em}homes,

		we want your \ch{Em\up{addC}}lives,

		we want the \ch{Em6}things you won't \ch{Em\up{addC}}allow us. 

		We won't use \ch{Em}guns, 

		we won't use \ch{Em\up{addC}}bombs

		We'll use the \ch{Em6}one thing we've got \ch{Em\up{addC}}more of --

		that's our \ch{Em}minds. \ch{Em\up{addC}} \hspace{20pt} \ch{Em6} \qquad \ch{Em}

	\end{SongVerse}

	\begin{SongVerse}

		\ch{A} Check your lucky numbers.

		\ch{E}That much money could drag you under, oh. \ch{F\#m}

		What's the point of being rich \ch{D}

		if you can't think what to do 
		
		with it 

		'cos \ch{D\up7}you're so bleeding thick?

	\end{SongVerse}

	\begin{SongVerse}

		
		\ch{A} \quad Oh, we weren't supposed to \ch{E}be -- 

		we learnt too much at school now 

		\ch{F\#m} we can't help but see 

		that the \ch{D}future that you've got mapped out 

		is \ch{D\up7} nothing much to shout about. 

	\end{SongVerse}

	\begin{SongVerse}

		
		\ch{G} \quad We're making a \ch{G\up{maj7}}move, 

		we're making it \ch{G\up 7}now. 

		We're coming out of the sidelines. 

		\ch{C}\quad Just put your \ch{C\up{maj7}}hands up -- it's a \ch{C\up7}raid \ch{C}... \ch{C\up7}yeah.

		We want your \ch{Em}homes,

		we want your \ch{Em\up{addC}}lives,

		we want the \ch{Em6}things you won't \ch{Em\up{addC}}allow us. 

		We won't use \ch{Em}guns, 

		we won't use \ch{Em\up{addC}}bombs

		We'll use the \ch{Em6}one thing we've got \ch{Em\up{addC}}more of --

		that's our \ch{Em}minds. \ch{Em\up{addC}} \hspace{30pt} \ch{Em6} \qquad \ch{Em}

	\end{SongVerse}

	\begin{SongVerse}
				\ch{A}Brothers, sisters, can't you \ch{E}see? 

		The future's owned by you and \ch{F\#m}me. 

		There won't be fighting in the \ch{D}street. 

		They think they've got us beat, 

		But \ch{D\up7}revenge is going to be so sweet. 
	\end{SongVerse}

	\begin{SongVerse}		

		
		\ch{G} \quad We're making a \ch{G\up{maj7}}move, 

		we're making it \ch{G\up 7}now. 

		We're coming out of the sidelines. 

		\ch{C}\quad Just put your \ch{C\up{maj7}}hands up -- it's a \ch{C\up7}raid \ch{C}... \ch{C\up7}yeah.

		We want your \ch{Em}homes,

		we want your \ch{Em\up{addC}}lives,

		we want the \ch{Em6}things you won't \ch{Em\up{addC}}allow us. 

		We won't use \ch{Em}guns, 

		we won't use \ch{Em\up{addC}}bombs

		We'll use the \ch{Em6}one thing we've got \ch{Em\up{addC}}more of --

		that's our \ch{Em}minds. \ch{Em\up{addC}} \hspace{30pt} \ch{Em6} \qquad \ch{Em}

	 \end{SongVerse}

\end{song}

\end{document}